\chapter{Tecnologías}
 En este capítulo se detalla todo lo relativo a los lenguajes de programación, los entornos de desarrollo y los frameworks elegidos para llevar a cabo este proyecto, así como sus características principales y la razón de su uso.
    
    \section{Lenguajes de programación}
    
    \subsection{TypeScript}
    TypeScript es un lenguaje de programación orientado a objetos(OO) el cual es superset de JavaScript. 
    Decimos que una tecnología es un superset de un lenguaje de programación, cuando puede ejecutar programas de la tecnología \cite{typescript}. En resumen, ejecutará el código como si fuese javaScript. \newline
        
    TypeScript se diferencia de JavaScript principalmente en que posee inferencia de tipos, es decir, está fuertemente tipado, además de algunas funcionalidades extra.
    \newline
        
    Este lenguaje se utiliza en el FrontEnd del proyecto, dado que el framework elegido para realizar esta parte es Angular, el cual se explicará en detalle más adelante.
        
    \subsection{HTML-5}
    HTML-5 (HyperText Markup Language) es la quinta revisión importante del lenguaje básico de la World Wide Web\cite{html}. Se trata de un lenguaje de marcación para la elaboración del contenido de las páginas web.
    Hoy en día es el lenguaje estándar que aceptan la gran mayoría de los navegadores a la hora de la construcción de las páginas web.
    \newline
     
    HTML-5 se diferencia de sus versiones anteriores en que incorpora nuevas etiquetas(section, article, header, footer etc...) con las cuales se busca mejorar y estandarizar la estructura de las páginas web además de otras actualizaciones como la mejora de los formularios o la inclusión de elementos de audio y vídeo.
    \newline
    
    Se ha optado por utilizar este lenguaje de marcación debido a su popularidad y a la inclusión de nuevas etiquetas que favorecen la lectura de la página web por parte de los navegadores.
    \newline
    
    
    \subsection{CSS-3}
    CSS (Cascading Style Sheets) es un lenguaje de diseño gráfico para definir y crear la presentación de un documento estructurado escrito en un lenguaje de marcado\cite{css}. Se utiliza en la mayoría de sitios web junto con html para generación de páginas web. De esta manera es mucho más sencillo generar páginas web, ya que, el diseño(CSS) se encuentra separado del contenido(HTML).
    \newline
    
    
    
    \subsection{Java 8}
    Java es un lenguaje de programación y una plataforma informática comercializada por primera vez en 1995 por Sun Microsystems \cite{java}. Proviene de los lenguajes C y C++, y sus aplicaciones pueden ser ejecutadas en cualquier JVM(Java Virtual Machine).
    \newline
    
    Para este proyecto se ha utilizado la version 8 porque esta versión es la que menos bugs tiene y la que mejora más la eficacia en el desarrollo y la ejecución de programas Java. \cite{java8} Además de estas razones, escogimos java por ser un lenguaje orientado a objetos ideal para desarrollas proyectos API-REST.
    
    \subsection{SQL}
    SQL es un lenguaje declarativo estándar internacional de comunicación dentro de las bases de datos que nos permite a todos el acceso y manipulación de datos en una base de datos \cite{sql}.
    \newline
    
    Se decidió utilizar este lenguaje debido a su uso en el SGDB(Sistema de Gestión de Base de Datos) que utilizamos en el proyecto, MySQL. Además de esto, SQL es ideal para trabajar con JPA(Java Persistence API) en el BackEnd del proyecto.
    
     \section{Entornos de desarrollo}
     \subsection{Visual Studio Code}
     Visual Studio Code es un editor de código fuente desarrollado por Microsoft para Windows, Linux y macOS. Incluye soporte para la depuración, control integrado de Git, resaltado de sintaxis, finalización inteligente de código, fragmentos y refactorización de código.\cite{vsc}.
    \newline
    
    En este proyecto se ha utilizado este entorno de desarrollo para gestionar un proyecto Angular, ya que este editor posee gran versatilidad a la hora de instalar plugins y gestionar diferentes lenguajes de programación de manera simultánea.
    
     \subsection{MySQL Workbench}
     Se trata de un programa para gestionar, diseñar y adminstrar bases de datos relacionales, utilizado en nuestro proyecto a la hora de manejar datos de manera local. \\
     \newline
     Se decidió utilizar debido a que se ha usado previamente en nuestros estudios de grado en diversas asignaturas de manera productiva, además de que posee una versión gratuita.
     
     \subsection{PhpMyAdmin}
     Al igual que MySQL Worbench se trata de una herramienta de gestión de bases de datos Mysql, pero con la diferencia que el acceso a esta herramienta es vía web, alojándose en un servidor.\\
     \newline
     Esta herramienta la utiliza Hostinger\cite{hostinger},  proveedor de alojamiento web donde se ha decidido alojar el proyecto para desplegarlo en la web.
     
     \subsection{GitHub}

     
     \subsection{BitBucket}
     
     
     \subsection{Eclipse}
     
     
     \subsection{OverLeaf}
     
     \section{Frameworks}
     