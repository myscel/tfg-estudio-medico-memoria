\chapter{Trabajo Futuro}
 Finalmente en este apartado nos gustaría resaltar algunas funcionalidades extra que, aunque no son imprescindibles, nos habría gustado tener tiempo de implementar:
 \newline
 
 \begin{itemize}
  \item\textbf{Gráficas comparativas en detalle.} \\
  La funcionalidad que se quedo en el tintero de mas importancia. Aunque el equipo médico aseguro repetidas veces que ellos funcionaban siempre con tablas de excel para el manejo de datos no cabe duda de que unas gráficas ilustrativas de la información recopilada en el estudio habrían sido útiles. Quizá no aportasen nada nuevo a este proyecto de investigación en concreto pero a futuro podrían permitir descubrir otros patrones en los pacientes o servir de base a investigaciones distintas. \\
  
  \item\textbf{Formularios dinámicos.} \\
  Una idea que surgió durante el desarrollo de los últimos detalles de la aplicación y habría ampliado muchísimo su potencia. Básicamente se planteo la idea de poder añadir, eliminar o modificar campos en los formularios de tal forma que la plataforma sería totalmente reutilizable para cualquier proyecto. Esto por supuesto supone un desafió a nivel de modelo de datos, la información pasaría a ser mas cómoda almacenada en JSON que en estructuras estadías, la comunicación entre componentes debería saber adaptarse, etc. En definitiva un trabajo complejo pero que nos habría encantado realizar. \\
  
  \item\textbf{Sistema de notificaciones.} \\
  Por último actualmente la comunicación entre investigadores y administradores se realiza de forma externa a la aplicación. Esto actualmente no supone un problema pues todos son compañeros de trabajo y tienen sus propios móviles para hablar en caso de no poder en persona, pero si a futuro se buscase ampliar el proyecto tendría que haber algún tipo de mensajería o sistema de notificaciones. 
  \newline
  
  El sistema serviría principalmente para que los investigadores hicieran peticiones de cambios al administrador, ya que ellos mismos no pueden tocar las citas ya completadas, y para que el administrador les comunicase errores o cambios detectados. En principio un sistema de mensajes cortos serían mas que suficiente pero al ponerlo en uso quizá se descubriesen otras herramientas de comunicación mas útiles o eficientes. \\

\end{itemize}